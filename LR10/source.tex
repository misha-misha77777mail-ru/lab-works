\documentclass{article}
\usepackage[russian]{babel}
\usepackage[utf8]{inputenc}
\usepackage{geometry}
\usepackage{mathtools}

\geometry{paperwidth=491.84pt,paperheight=9.25in}
\geometry{lmargin=86.383pt,bmargin=80pt,rmargin=86.383pt,tmargin=80.55pt}

\title{Page 31}
\author{Власко М. М.}
\date{November 2023}

\begin{document}
\pagenumbering{\textsl{31}}
{\setlength{\parindent}{0cm}
\textsf{\textbf{Определение 7.}}
\textit{Группы, состоящие из k элементов в каждой и отличающиеся друг от друга по крайней мере одним элементом, называются сочетаниями.}
}

Например, группы \{1, 2\}, \{1, 3\} и \{2, 3\} образуют все сочетания из натуральных чисел 1, 2, 3 по два из них.

Число всех сочетаний из \textit{n} элементов по \textit{k} элементов в каждом обозначется \(C_n^k\).
\newline
{\setlength{\parindent}{0cm}
   \textsc{\textbf{ТЕОРЕМА 3. }}\textit{Имеет место формула}
}
\[ C_n^k = \frac{A_n^k}{P_k}, \tag{1.15}\]
{\setlength{\parindent}{0cm}
   \textit{т. е.}
}
\[ C_n^k = \frac{n(n - 1)...(n - k + 1)}{n!}. \tag{1.16}\]
{\setlength{\parindent}{0cm}
   \textsc{\textbf{СЛЕДСТВИЕ. }}\textit{Справедлива формула}
}
\[ C_n^k = \frac{n!}{k!(n - k)!}. \tag{1.17}\]

{\setlength{\parindent}{0cm}
   \textsc{Доказательство.}
Если в каждом сочетании из \textit{n} элементов по \textit{k} (их всего \(C_n^k\)) сделать всевозможные перестановки
   его элементов (число таких перестановок равно \(P_k\)), то получатся размещения из \textit{n} элементов по
   \textit{k}, причём таким способом получаются все размещения из \textit{n} элементов по \textit{k}, и притом по
   одному разу. Поэтому \[ C_n^kP_k = A_n^k,\] откуда и следует формула (1.15).
   Формула (1.16) получается из формул
   (1.10), (1.14), (1.15).
}
\par{Если умножить числитель и знаменатель дроби, стоящей в правой части формулы (1.16), на \((n - k)\)!, то
получится формула (1.17).}
{\setlength{\parindent}{0cm}
   \textsc{\textbf{ТЕОРЕМА 4. }}\textit{Имеет место формула}
   \[ C_n^k = C_n^{n - k}, k = 0, 1, 2, \dots, n, \tag{1.18}\]
\textit{где} \(C_n^0 \stackrel{\text{def}}{=} 1\).
}
\newline
{\setlength{\parindent}{0cm}
   \textsc{\textsf{Доказательство.}}
Формулу (1.18) легко получить непосредственно из определения сочетаний: если из \textit{n} элементов выбрать какую-либо
   группу (сочетание), состоящую из \textit{k} элементов, то останется группа (сочетание) из \(n - k\) элементов, при
   этом таким способом получаются все сочетания из \textit{n} элементов по \(n - k\) элементов и по одному разу.
}

\centering
-----------------------
\end{document}